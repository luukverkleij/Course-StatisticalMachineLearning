\documentclass{article}
\usepackage{listings}

\begin{document}
\section{Assigment1.py}
\begin{lstlisting}[language=python]
import numpy as np
import matplotlib.pyplot as plt

'''
Exercise 1: 
'''
def f(x):
return 1 + np.sin(6 * (x - 2))

#t = np.arange(0.0, 1.0, 0.01)
#t2 = np.arange(0.0, 1.0, 0.1)
#t3 = np.arange(0.0, 1.0, 0.025)

np.random.seed(0)

def createNoise(n):
return np.random.normal(0, 0.3, n)


def createSignal(n):
t = np.arange(0.0, 1, 1 / n)
return (t, [x + y for x, y in zip(createNoise(n), [f(x) for x in t])])

t10 = np.arange(0.0, 1, 0.11111111111111)
print(len(t10))
s10 = (t10, [x + y for x, y in zip(createNoise(10), [f(x) for x in t10])])
s40 = createSignal(40)
s100 = createSignal(100)


#testset = [x + y for x, y in zip(createNoise(100), [f(x) for x in t])]
#dataset = [x +  y for x, y in zip(createNoise(10), [f(x) for x in t2])]
#dataset40 = [x + y for x, y in zip(createNoise(40), [f(x) for x in t3])]


def sumx(x, i):
j = 0
s = 0
while(j < len(x)):
s += x[j]**i
j += 1
return s


def sumy(x, t, i):
j = 0
s = 0
while(j < len(x)):
s += t[j] * x[j]**i
j += 1
return s


def kdelta(i, j):
# works due to pythons representation of true (1) and false (0)
return i == j

'''
Exercise 1_2: E(w) = 0.5 * SUM(N, n=1, {y(xn; w) − tn}^2)
2

'''
def PolCurFit(x, t, m):
# Exercise 2
# D_n = (x, t)
A = np.fromfunction(lambda i, j: sumx(x, i + j), (m + 1, m + 1), dtype=int)
T = np.fromfunction(lambda i, j: sumy(x, t, i), (m + 1, 1), dtype=int)
return np.linalg.solve(A, T)

'''
Exercise 1_5: E˜ = E + λ/2 * SUM(M, j=0, w[j]^2)
'''
def PolCurFit2(x, t, m, λ):
# Exercise 5
# Sum_j=0^M Aij * wj = Tj
# Based on exercise sheet 2, exercise 2.3
A = np.fromfunction(
lambda i, j: sumx(x, i + j) + λ * kdelta(i, j),
(m + 1, m + 1),
dtype=int)
T = np.fromfunction(lambda i, j: sumy(x, t, i), (m + 1, 1), dtype=int)
return np.linalg.solve(A, T)


def poly(vec, x):
s = 0
j = 0
while(j < len(vec)):
s += x**j * vec[j]
j += 1
return s

def error_rms(w, ts, ds):
E = 0
n = len(ts)
for j in range(0, n):
E += (poly(w.flatten(), ts[j]) - ds[j])**2
rms = np.sqrt(E/n)

return rms


'''
Functions to execute the exercises.
'''

def exercise1_1():
plt.plot(s100[0], f(s100[0]), label=("f"))
plt.plot(s10[0], s10[1], 'ro', label=("dataset 10"))
plt.legend()
plt.show()

def exercise_a(ts, ds):
for i in range(0, 10):
w = PolCurFit(ts, ds, i)
x = np.linspace(0, 1, 100)
y = poly(w.flatten(), x)
plt.plot(x, y, linestyle="dashed", label=("M=" + str(i)))
plt.plot(ts, ds, 'ro')
plt.plot(s100[0], f(s100[0]), linewidth=1.5, label=("f"))
plt.legend(bbox_to_anchor=(0., 0.85, 1., 0), loc=3,
ncol=4, mode="expand", borderaxespad=0.)
plt.gca().set_ylim([-1,4])

plt.show()

def exercise_b(ts, ds):
e_train = []
e_test = []
for i in range(0, 10):
w = PolCurFit(ts, ds, i)
e_train += [error_rms(w, ts, ds)]
e_test += [error_rms(w, s100[0], s100[1])]


plt.plot(np.arange(0, 10, 1), e_train, label=("training set"))
plt.plot(np.arange(0, 10, 1), e_test, label=("test set"))
plt.legend()
plt.show()


def exercise1_3a():
exercise_a(s10[0], s10[1])

def exercise1_3b():
exercise_b(s10[0], s10[1])

def exercise1_4a():    
exercise_a(s40[0], s40[1])

def exercise1_4b():
exercise_b(s40[0], s40[1])


def exercise1_5():
for i in range(1, 10):
w = PolCurFit2(s10[0], s10[1], i, np.exp(-18))
x = np.linspace(0, 1, 100)
y = poly(w.flatten(), x)
plt.plot(x, y, linestyle="dashed", label=("M=" + str(i)))
plt.plot(s10[0], s10[1], 'ro')
plt.plot(s100[0], f(s100[0]), linewidth=1.5, label=("f"))
plt.legend(bbox_to_anchor=(0., 0.85, 1., 0), loc=3,
ncol=4, mode="expand", borderaxespad=0.)
plt.show()

def exercise1_5b():
e_train = []
e_test = []

rng = range(-40, 1)

for lbd in rng:
w = PolCurFit2(s10[0], s10[1], 9, np.exp(lbd))
e_train += [error_rms(w, s10[0], s10[1])]
e_test += [error_rms(w, s100[0], s100[1])]

plt.plot(rng, e_train, label="Training Set")
plt.plot(rng, e_test, label="Test Set")

plt.ylabel("Error RMS")
plt.xlabel("ln(lambda)")

plt.legend()

plt.show()

def exercise1_5c():
polys = []
rng = range(-20, 1)

for lbd in rng:
w = PolCurFit2(s10[0], s10[1], 9, np.exp(lbd))
polys += [w.flatten()]

#print(polys)
for i in range(4,10):
plt.plot(rng, [x[i] for x in polys], label=str(i))

plt.legend()
plt.show()


exercise1_5c()

'''print(np.exp(10))

for i in range(1, 10):

w = PolCurFit2(t2, dataset, i, np.exp(-18))
x = np.linspace(0, 0.9, 100)
y = poly(w.flatten(), x)
plt.plot(x, y, label=("M=" + str(i)))



plt.plot(t, f(t))
plt.plot(t2, dataset, 'ro')
plt.legend()
plt.show()
'''

\end{lstlisting}

\section{Assigment2.py}
\begin{lstlisting}[language=python]
from mpl_toolkits.mplot3d import Axes3D
from matplotlib import cm
from matplotlib.ticker import LinearLocator, FormatStrFormatter
import matplotlib.mlab as mlab
import matplotlib.pyplot as plt
import numpy as np


def h(x, y):
return 100 * (y - x**2)**2 + (1 - x)**2


def dhx(x, y):
return 2 * (-1 + x + 200 * x**3 - 200 * x * y)


def dhy(x, y):
return 200 * (y - x**2)

def descent_next(w_x, w_y, epsilon):
return ((-epsilon * dhx(w_x, w_y)), (-epsilon * dhy(w_x, w_y)))


def descent(w_x ,w_y, epsilon, precision):
while True:
dn = descent_next(w_x, w_y, epsilon)
w_x = w_x + dn[0]
w_y = w_y + dn[1]
if np.sqrt((dn[0]/epsilon)**2 + (dn[1]/epsilon)**2) < precision:
break    

return (w_x, w_y)

def plot_descent(w_x, w_y, epsilon, precision, color="blue"):
start_x = w_x
start_y = w_y

xs = [w_x]
ys = [w_y]
tel = 0
while True:
tel+=1
dn = descent_next(w_x, w_y, epsilon)
w_x = w_x + dn[0]
w_y = w_y + dn[1]

##        if not xs or np.sqrt((w_x - xs[-1])**2 + (w_y - ys[-1])**2) > 0.05:
xs += [w_x]
ys += [w_y]        

if np.sqrt((dn[0]/epsilon)**2 + (dn[1]/epsilon)**2) < precision:
break

xs += [w_x]
ys += [w_y]
print(tel)
#plt.scatter(xs, ys, color=color)
plt.plot(xs, ys, linewidth=2, color=color, label="(" + str(start_x) + "," + str(start_y) + ")")

def plot_surface():
fig = plt.figure()
ax = fig.gca(projection='3d')
x = np.arange(-2.0, 2.0, 0.25)
y = np.arange(-1.0, 3.0, 0.25)
X, Y = np.meshgrid(x, y)
Z = h(X, Y)


surf = ax.plot_surface(X, Y, Z, rstride=1, cstride=1, cmap=cm.coolwarm, linewidth=0, antialiased=False)

ax.set_zlim(-2, 3000)

ax.zaxis.set_major_locator(LinearLocator(10))
ax.zaxis.set_major_formatter(FormatStrFormatter('%.02f'))

fig.colorbar(surf, shrink=0.5, aspect=5)

def plot_contour():
x = np.arange(-2.0, 2.0, 0.001)
y = np.arange(-1.0, 3.0, 0.001)
X, Y = np.meshgrid(x, y)
Z = h(X, Y)
plt.figure()
levels = [0,0.1,1,10,100,400,1200]
CS = plt.contour(X, Y, Z, levels, colors=('k'))
plt.clabel(CS, inline=1, fontsize=10)
plt.title('Simplest default with labels')
CB = plt.colorbar(CS, shrink=0.8, extend='both')
l, b, w, h1 = plt.gca().get_position().bounds
ll, bb, ww, hh = CB.ax.get_position().bounds
CB.ax.set_position([ll, b + 0.1*h1, ww, h1*0.8])

def assigment2_1():
plot_surface()
plt.show()

def assigment2_4():
plot_contour()

#plot_descent(-2, -1, 0.0015, 0.000000001, "blue")
plot_descent(0, 3, 0.0015, 0.000000001, "green")
plot_descent(2, 3, 0.0015, 0.000000001, "red")
plt.legend(bbox_to_anchor=(0., 0.25, 1., 0))
plt.show()

assigment2_4()

\end{lstlisting}
\end{document}
